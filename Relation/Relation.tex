%============================== Setting teh Document=============================================

\documentclass[aspectratio=169]{beamer}
\usepackage[italian]{babel} 
\usepackage[utf8]{inputenc} 
\usepackage[T1]{fontenc}
\usepackage{graphicx} 
\usepackage{hyperref}
\usepackage{xcolor}
\usepackage{amsmath,amssymb,lmodern}
\usetheme{AnnArbor}

% in way to comment a block
\long\def\/*#1*/{}

%==========================Set the foot line========================================================
\makeatletter
\setbeamertemplate{footline}
{
	\leavevmode%
	\hbox{%
		\begin{beamercolorbox}[wd=.333333\paperwidth,ht=2.25ex,dp=1ex,center]{author in head/foot}%
			\usebeamerfont{author in head/foot}\insertsection
		\end{beamercolorbox}%
		\begin{beamercolorbox}[wd=.333333\paperwidth,ht=2.25ex,dp=1ex,center]{title in head/foot}%
			\usebeamerfont{title in head/foot}\insertsubsection
		\end{beamercolorbox}%
		\begin{beamercolorbox}[wd=.333333\paperwidth,ht=2.25ex,dp=1ex,right]{date in head/foot}%
			\usebeamerfont{date in head/foot}\insertshortdate{}\hspace*{2em}
			\insertframenumber{} / \inserttotalframenumber\hspace*{2ex} 
	\end{beamercolorbox}}%
}%
\vskip0pt%

\makeatother



\title{Il VHF marino e le procedure di radio telefonia} 
\author{Francesco Rombaldoni\\
Alias Rombo} 
\date{}

\institute{Università degli Studi di Urbino "Carlo Bo"} 
\logo{\includegraphics[width=15mm]{Uni}}

\setbeamercovered{dynamic}

%===========================================Document starting===============================================
\begin{document}
	
	% Cover Page
	\begin{frame} 
		\maketitle 		
	\end{frame}
	
	% Index Page
	\begin{frame}
		%\index{generate}
		\tableofcontents
	\end{frame}

	\section{Premessa}
	
	\begin{frame}{Premessa}
		%\framesubtitle{Parte 1}
		\textbf{L'obiettivo di questa relazione è di esporre il funzionamento, l'utilizzo e la manutenzione degli apparati radio ricetrasmittenti adottati per le radiocomunicazioni in ambito nautico.}\\
		\bigskip
		Per fare ciò la relazione sarà suddivisa in due parti:\\
		\begin{itemize}
			\item La prima parte verterà sulla teoria fisica di base degli apparati radio, al termine della quale si comprenderanno espressioni linguistiche del tipo: \emph{" La maggior parte delle trasmissioni nautiche sono effettuate utilizzando apparati VHF che modulano il segnale in frequenza"}.\\
			\item La seconda parte verterà invece sull'utilizzo e la manutenzione degli impianti ricetrasmittenti di bordo, riservando attenzione alle "buone pratiche" e alle procedure per una comunicazione efficace, al termine della parte si sarà in grado di stabilire una comunicazione con un'altra unità marina o con la capitaneria di porto.
		\end{itemize}
	\end{frame}

	\begin{frame}{Premessa}
		\framesubtitle{Argomenti della relazione}
		\textbf{Argomenti della prima parte.}
		\begin{itemize}
			\item Proprietà elettromagnetiche della corrente alternata.
			\item Periodo, frequenza e lunghezza d'onda.
			\item Modulazione di ampiezza e la Modulazione di frequenza.
			\item Le bande radio.
			\item La radio e l'impianto ricetrasmittente.
		\end{itemize}
	\end{frame}

	\begin{frame}{Premessa}
		\framesubtitle{Argomenti della relazione}
		\textbf{Argomenti della seconda parte.}
		\begin{itemize}
			\item La banda nautica.
			\item I canali VHF.
			\item L'alfabeto internazionale. 
			\item La classificazione dei messaggi. 
			\item Trasmissione e ricezione.
		\end{itemize}
	\end{frame}

	\section{Contesto}
	\begin{frame}{Contesto}
		\framesubtitle{Parte 1}
		Per trasmettere informazioni, senza disporre di un collegamento fisico tra chi trasmette e chi riceve occorre far transitare su un mezzo non materiale.\\
		\bigskip
		\textbf{Nella trasmissione senza fili il mezzo materiale è costituito da \emph{onde radio}, ovvero da \emph{perturbazioni elettromagnetiche dello spazio}.}\\
		\bigskip
		Le onde radio possono veicolare le informazioni tra una stazione detta \emph{trasmittente} e una stazione detta \emph{ricevente}.\\
	\end{frame}

	\begin{frame}{Contesto}
		\framesubtitle{Parte 2}
			\textbf{Per trasmettere informazioni occorre:}\\
		\begin{itemize}
			\item Generare un'onda radio, detta \emph{onda portante} (quella che veicola le informazioni).
			\item Sovrapporre all'onda portante le informazioni che si desidera veicolare, l'onda che si ottiene è definita \emph{un'onda modulata}.
		\end{itemize} 
		\bigskip
		\textbf{Per ricevere informazioni occorre:}
		\begin{itemize}
			\item Captare l'onda radio trasmessa.
			\item Estrarre dall'onda modulata le informazioni, ovvero \emph{demodulare l'onda}.
		\end{itemize}
	\end{frame}

	\section{Proprietà elettromagnetiche della corrente alternata}
	
	\/*
	\begin{frame}{Introduzione}
		\framesubtitle{Campo elettromagnetico}
		Nel 1831 Faraday pubblica un resoconto di una serie di esperimenti nei quali, con modalità opportune, era riuscito a \emph{indurre in un circuito metallico una corrente elettrica} 
		facendo muovere un magnete rispetto al circuito.\\
		\bigskip
		Nel 1890 Lorentz studia l'interazione tra una carica elettrica e un campo magnetico, comprendendo che il campo magnetico esercita una forza sulla carica elettrica secondo questa espressione:\\
		\centering
		\textbf{\textcolor{red!80}{$\vec{F}$ = q $\vec{v}$ X $\vec{B}$}}\\
		\raggedright
		questo testo invece non dovrebbe essere centrato
		prova di scrittura
	\end{frame}*/

\begin{frame}{Introduzione}
	\framesubtitle{Campo elettromagnetico}
	Nel 1831 Faraday pubblica un resoconto di una serie di esperimenti nei quali, con modalità opportune, era riuscito a \emph{indurre in un circuito metallico una corrente elettrica} 
	facendo muovere un magnete rispetto al circuito.\\
	\bigskip
	Il fenomeno esaminato da Faraday sarà spiegato nel 1890 dagli studi di Lorentz riguardo le interazioni del campo elettrico e del campo magnetico su una carica elettrica; gli studi dimostrano infatti che se una carica elettrica è situata in un campo elettrico e in un campo magnetico sovrapposti, allora su di essa agisce una forza proporzionale: dal valore della carica, dalla sua velocità e dall'intensità dei due campi. 
\end{frame}

\begin{frame}{Forza di Lorentz}
	\framesubtitle{Campo elettromagnetico}
	
\end{frame}


\end{document}